The first step to run GPUSPH is to install the \nvidia\ company's CUDA
compilers and libraries (directions given below). CUDA is an extension
of the \cpp\ language to allow \cpp\ to talk to the graphics card.

The second step is to install the open source software, Open Dynamics
Engine, which simulates rigid body dynamics. This library is used for
any rigid objects that move, such as floating objects or objects moved
by fluid flow.

The third step is to obtain, compile and run GPUSPH.

\section{Installing CUDA}

Ensure that your computer has an \nvidia\ graphics card that is CUDA
enabled. The \nvidia\ website has a list of all the CUDA-enabled graphics
cards: \url{www.nvidia.com/object/cuda_gpus.html}. You can check whether
CUDA is already installed on your machine by launching the command:
\begin{shellcode}
nvidia-smi
\end{shellcode}
from the terminal. If CUDA is installed it will give you information
on the current state of the NVIDIA graphics card(s) on the machine.\\

\textbf{Caution}: At the present time, you must have a card with at least Compute
Capability of 2.0 to run GPUSPH.
\\

\textbf{Remark}: 
Regarding the choice of the GPU, the more streaming 
processors and the more video memory on the card, the better. 
Also, the higher the Compute Capability, the more
CUDA language features can be used on the card. 
\todo{Update this}
Currently the Kepler
K40 has the most memory (12~GB), the most processors (${}>2600$), and
the highest Compute Capability (3.5) for scientific work. High end
gaming cards, such as the GTX Titan, have a similar numbers of
processors and memory, but with less computational features, such as
error-correcting memory. While they are not quite as robust for
scientific work, they are cheaper. On a MacBook Pro laptop, circa 2012,
the graphics card is a GeForce GT 650M, with 384 processors with 1024~MB
of VRAM with a Compute Capability of 3.0; more than enough to do
significant parallel computing.\\

The GPU programming language CUDA can be obtained from the \nvidia\ website,
CUDA Zone. The CUDA Toolkit and CUDA Software Development Kit (SDK)
need to be installed for your operating system along with the video
driver. These packages include the CUDA compiler \cmd{nvcc}, which is
needed to develop executable code, and the graphics card driver that
allows your program to access the GPU card. \\

Download the relevant driver for your machine from:\\
\url{http://www.nvidia.com/Download/index.aspx?lang=en-us} \\
and the CUDA toolkit from:\\
\url{https://developer.nvidia.com/cuda-downloads}\\
Follow the instructions provided by \nvidia\ for the installation.\\


To ensure that all is installed correctly and working, you should
compile and run the SDK examples, which include many programs that
illustrate the capabilities of CUDA and the GPU; for example, \nvidia's
sorting program \cmd{radixSort} is used by GPUSPH to organize the
neighbor list. Some interesting SDK programs are \cmd{fluidsGL} and
\cmd{particles}. To compile the SDK programs, after the SDK is
installed, go to \url{/Developer/GPU Computing/C} and (on a unix/linux
or mac machine), type \cmd{make} on a terminal window command line. This
should create a directory of executable examples located within the C
directory called bin/darwin/release for the mac and bin/linux/release
for a linux machine. In this directory, type \cmd{./fluidsGL} to run
the \cmd{fluidsGL} example. You should see a green window open on your
desktop. Use the mouse to stir up the fluid. The example program
Particles is worth playing with as well, as it provided a basis for
developing GPUSPH.\\



\section{Installing the Open Dynamics Engine}
\todo{Replace with CHRONO}
The website for the Open Dynamics Engine is \url{http://www.ode.org},
with links to the Source Forge repository to download the code. This
needs to be installed to run GPUSPH. If you use moving rigid bodies in
your problems, you will need the manual (available from the link above
but here it is anyway:
\url{http://ode-wiki.org/wiki/index.php?title=Manual:_Introduction}) to
assist in the writing of your own problem.

Please note that ODE should be compiled in single-precision mode.

\todo{How to install relevant packages on common distributions
(Debian/Ubuntu, Arch, Fedora/RedHat).}

\section{Installing GPUSPH}

The GPUSPH source code is hosted on \href{http://github.com}{GitHub}.
The project's GitHub page is \url{http://github.com/GPUSPH/gpusph}.

To obtain the GPUSPH code, you can either use the \cmd{git} revision
control system, or download a \cmd{.zip}ped archive of a specific
version. This manual refers to \currentver\ of GPUSPH.

If you have \cmd{git} installed, you can use
\begin{shellcode}[escapeinside=\{\}]
git clone https://github.com/GPUSPH/gpusph.git
cd gpusph
git checkout v{\version}
\end{shellcode}
to get \currentver\ specifically. Otherwise, download the \cmd{.zip}ped
archive from \url{http://github.com/GPUSPH/gpusph/archive/v\version.zip},
and then
\begin{shellcode}[escapeinside=\{\}]
unzip v{\version}.zip
cd gpusph-{\version}
\end{shellcode}
(you may remove \cmd{v\version.zip} afterwards).

Within the top directory, you can find the \cmd{Makefile}, a \cmd{src}
directory (holding the main GPUSPH source), a \cmd{scripts} directory
(holding various auxiliary scripts), a copy of the license, settings to
produce internal documentation with Doxygen, and a sample Digital
Elevation Model (DEM) data file.

The most interesting source files in \cmd{src} are the \cmd{Problem}s.
A few sample problems are shipped with GPUSPH, showing how to employ
specific features. You can get a list of the available problems by
running
\begin{shellcode}
make list-problems
\end{shellcode}

To build and test GPUSPH, you can run
\begin{shellcode}
make test
\end{shellcode}
which should automatically detect your configuration, such as the
compute capability of your GPU as well as the availability of optional
libraries such as MPI (for mulit-node support) or HDF5 (to read HDF5SPH
data files).

When the building completes, you will have some new directoryes
(\cmd{build} and \cmd{dist}) and a \cmd{GPUSPH} soft link to the
compiled binary. \cmd{make test} will also automatically run
\cmd{./GPUSPH} for you.

After building, simply runnning \cmd{./GPUSPH} will run the program
again.


\subsection{Choosing the \cmd{Problem} and other options}

You can test a different problem by using:
\begin{shellcode}
make problem=OtherProblem test
\end{shellcode}
where \cmd{OtherProblem} is the name of a different problem. You can get
a list of available problems with \cmd{make list-problems}.

There are a number of other options available. A complete list of the
options and their description can be obtained by running \cmd{make
help-options}. All options (with the exception of \cmd{plain} and
\cmd{echo}) are persistent across compilations, so they can be set once
with \cmd{make option=value}, and subsequent executions of \cmd{make}
will remember the \cmd{value} set.

\todo{Describe \cmd{make} options}
The \cmd{make} options are listed below:
\begin{itemize}
\item \cmd{target_arch} - if set to 32, force compilation for 32 bit architecture
\item \cmd{problem} - Name of the problem.
\item \cmd{dbg} - 0 no debugging, 1 enable debugging
\item \cmd{compute} - 11, 12, 13, 20, 21, 30, 35, etc: compute capability to compile for (default: autodetect)
\item \cmd{fastmath} - Enable or disable fastmath. Default: 0 (disabled)
\item \cmd{mpi} - 0 do not use MPI (no multi-node support), 1 use MPI (enable multi-node support). Default: autodetect
\item \cmd{hdf5} - 0 do not use HDF5, 1 use HDF5, 2 use HDF5 and HDF5 requires MPI. Default: autodetect
\item \cmd{verbose} - 0 quiet compiler, 1 ptx assembler, 2 all warnings
\item \cmd{plain} - 0 fancy line-recycling stage announce, 1 plain multi-line stage announce
\item \cmd{echo} - 0 silent, 1 show commands
\end{itemize}
To view your current make options type \cmd{make show} instead of make.

\section{Example Problems}

Simulations in GPUSPH are defined in terms of \cmd{Problem}s. Some
example problems are provided with GPUSPH itself, to illustrate the
basics of problem design, and how to use the fundamental building blocks
provided by GPUSPH. Such building blocks include a variety of
geometrical shapes to describe the (fixed) solid boundaries of the
domain, as well as a number of objects that move following prescribed
laws, such as gates, pistons and paddles.

These objects are designed to offer great flexibility in their use, far
beyond what is shown in the sample problems. This flexibility should
allow you to create very complex simulations by combining the objects
appropriately.

\iffalse
GPUSPH has options for specified moving objects, which are used to make
piston and paddle wavemakers and a moving gate. These objects are
comprised of particles that are distinguished by identifying their type
as GATEPART, PISTONPART, and PADDLEPART. (Water is distinguished by
FLUIDPART and fixed boundary particles are of type BOUNDPART.) The
distinction between GATEPART and PISTONPART is that the particles of the
GATE are moved by providing an arbitrary (possibly time-varying)
velocity vector in the problem's callback function and a PISTONPART
particle is moved by providing a displacement for the vertical piston in
(only the) x direction with time, again via the callback function.
\else
\todo{blurb about the various moving object types shold be moved
elsewhere}
\fi

The number of particles used in the test problems is deliberately taken
as a small number, simply to allow for fast execution times even on
older hardware. One of the first tests to try is to increase the
resolution by reducing the size of the particles. For example,
by~reducing the particle size from the default of~$0.025$m to the
smalle~$0.02$m, \cmd{DamBreak3D} would run with $21,252$ particles
instead of the default $10,664$.

This can be done in two ways. A permanent change comes about by editing
the problem file (e.g. \cmd{DamBreak3D.cc}) and changing the value
passed as argument of \cmd{set_deltap()} (e.g., replace
\cmd{set_deltap(0.025f);} with \cmd{set_deltap(0.02f);}. The second way
is to specify the particle size at runtime using the appropriate command
line option (described below): e.g. \cmd{./GPUSPH -{}-deltap 0.02}.

\subsection{DamBreak3D}

\cmd{DamBreak3D} is a case originally used by
\cite{gomez-gesteira_using_2004}
for testing a prototype version of SPHysics. It is based on some
experiments done by \cite{arnason_interactions_2005} at the University of Washington.
We assume an instantaneous breaking dam and the resulting flow impinging
onto a rectangular object. The whole problem is contained within a
bounding box, which extends $1.6$m in length ($x$ axis), $0.67$m in
width ($y$ axis), and $0.4$m in height. This is the experimental box.
The fluid behind the dam is a rectangular box of water at one end of the
tank at time equal to zero. The dam is assumed to break instantaneously
so that the column of water, confined on three sides, collapses into the
tank. In the tank there is a vertical rectangular object -- the
collapsing water column impacts on the tank and then flows up the front
face of the object and around the sides. Finally the water hits the back
wall of the tank. A screenshot of the simulation at time $0.6s$ is provided
in the Figure \ref{fig:DamBreak3D}.

\begin{figure}[h]
  \begin{center}
    \includegraphics[scale=0.3, trim={100 100 100 100},clip]{fig/damBreak3D_screenshot.png}
    \caption{Screenshot of the DamBreak3D simulation at time $0.6s$.}\label{fig:DamBreak3D}   
  \end{center}
\end{figure}

\subsection{DamBreakGate}

In most laboratory experiments of dam breaks, the dam takes a certain
amount of time to move out of the way. The example problem
\cmd{DamBreakGate} illustrates the use of moving boundary particles of
the type GATEPART. The problem is set up the same way as the
\cmd{DamBreak3D} case, but there is a moving gate that is raised
vertically with a linearly varying velocity. In this case, the gate will
move with a velocity that is zero when the problem starts and that
linearly increases with time until the gate is outside the domain. The
effect on the dam break is that the escaping water is affected by the
gage motion. (See \cite{crespo_modeling_2008}'s SPH modeling of
\cite{janosi_turbulent_2004}'s experiment, where a moving gate was important.)

The moving gate is created by defining its geometry with particles
denoted as GATEPART particles and the \cmd{mb_callback} function, which
is used for the \textbf{m}oving \textbf{b}oundaries.

A screenshot of the simulation at time $0.8s$ is provided
in the Figure \ref{fig:DamBreakGate}.

\begin{figure}[h]
  \begin{center}
    \includegraphics[scale=0.35, trim={100 100 100 100},clip]{fig/damBreakGate_screenshot.png}
    \caption{Screenshot of the DamBreakGate simulation at time $0.8s$.}\label{fig:DamBreakGate}   
  \end{center}
\end{figure}

\subsection{OpenChannel}

This problem represents an instantaneous start up of a highly viscous
and dense fluid flow in an open channel on a $9\deg$ slope. The
channel is rectangular in cross-section ($1$m wide and $0.7$m deep) and
the computed length of the infinitely long channel is $2$m. The side
walls are fixed (Leonard-Jones boundary force) while the computational
ends of the domain are periodic, so that a particle leaving the
downstream end of the model domain enters the upstream end at the same
place, $2$m upstream.

The periodic boundary here is used in the $x$ direction, although
boundaries in other problems can be periodic in the other directions as
well. The key parameter in the problem statement is
\cmd{m_simparams.periodicbound}, which can be set to any combination of
\cmd{PERIODIC_X}, \cmd{PERIODIC_Y}, \cmd{PERIODIC_Z} to indicate
peridocity along each of the axes.
Figure \ref{fig:OpenChannel} shows the shape of the velocity field in 
the channel after time convergence.
\begin{figure}[h]
  \begin{center}
    \includegraphics[scale=0.35, trim={0 100 0 0},clip]{fig/openChannel_screenshot.png}
    \caption{Screenshot of the OpenChannel simulation after time convergence.}\label{fig:OpenChannel}
  \end{center}
\end{figure}

\subsection{WaveTank}

WaveTank uses a moving boundary to create a paddle wavemaker at one end
of a wave tank with a sloping bottom (bottom slope is $4.2364\deg$). The
wavemaker motion is controlled by the \cmd{mb_callback} function. In
this case, the length of the paddle is $1.0$m and the paddle pivots
about an origin \cmd{m_origin}; here, the pivot is located $0.1344$m
below the bottom and $0.13$m from the front wall of the tank. To specify
the paddle motion, the angular frequency of the motion ($2 \pi/T$, where
$T=1$s is the wave period), and the wave paddle stroke at the water
surface ($S=0.1$m) are given in the variables \cmd{mb_omega} and
\cmd{mb_amplitude}. To change the stroke and the frequency of the wave
paddle, you must change these variables in the problem file,
\cmd{WaveTank.cc}.
Figure \ref{fig:WaveTank} shows a screenshot of the simulation at time $9s$.

\iffalse
\begin{figure}[h]
\centering{%
\includegraphics[width=0.63\textwidth]{paddle.png}%
}
\caption{Schematic of the wave paddle for \cmd{WaveTank.cc}}
\end{figure}
\else
%\todo{paddle picture}
\fi

\begin{figure}[h]
  \begin{center}
    \includegraphics[scale=0.35, trim={0 200 0 200},clip]{fig/waveTank_screenshot.png}
    \caption{Screenshot of the WaveTank simulation at time $9s$.}\label{fig:WaveTank}   
  \end{center}
\end{figure}

\subsection{SolitaryWave}

SolitaryWave is similar in set up to the WaveTank example, except that a
piston moving boundary is used. The motion of a vertical plate is
determined by the method of \cite{goring_tsunamis_1979}, available in PDF format
from \cmd{http://caltechkhr.library.caltech.edu/50/}. The full
excursion (stroke) of the paddle is the variable \cmd{S}.

\subsection{Seiche}

The Seiche problem is to examine the influence of shaking on a
rectangular container of size: $\ell = 0.707$m, $w = \ell/2$, and depth,
$H = 0.5$m. The purpose of the example is to illustrate the ability to
vary gravity in a problem. As the problem starts, there is water in the
container. After $0.3$s, gravity is modified by adding a component in
the $x$ direction, such that the total gravity vector is
\cmd{m_physparams.gravity = make_float3(3.*sin(9.8*(t-m_gtstart)), 0.0,
-9.81f);}, which means that the container is shaken with a sinusoidal
motion with angular frequency of $9.8\text{s}^-1$ (period${} = 0.64$s),
with a magnitude of $3\text{m}/\text{s}^2$ until time \cmd{m_gtend=3.0}
is reached, when the gravity vector once again returns to the vertical
acceleration of gravity. After this time, the seiching motion starts to
decrease in amplitude.

\iffalse
\begin{figure}[h]
\centering{%
\includegraphics[scale=0.5]{Seiche.png}%
}
\caption{Resonant seiching in a rectangular domain showing the results
of a time varying gravity in the problem, \cmd{Seiche.cc}. Here the tank has
been shaking side to side at the resonant frequency of $0.638$s. The
color coding is for the pressure in the fluid.}
\end{figure}
\else
\todo{seiche picture}
\fi

The variation of gravity with time (and any stop (\cmd{m_gtend}) and
start times) is prescribed in a user-supplied (in the problem)
\cmd{g_callback} function.

\subsection{TestTopo}

This is an example showing how to use GPUSPH's support for Digital
Elevation Models (DEMs). It loads the topography of the bottom of the
domain from a file called \cmd{half_wave0.1m.txt}, shipped with GPUSPH.
A different DEM can be used, by either changing the name in the source
\cmd{TestTopo.cc} file, or by providing the new name as argument to the
\cmd{-{}-dem} command-line option to GPUSPH.

\todo{TestTopo picture}

\section{GPUSPH Command Line Options}\label{options}

When running from the command line, there are several options available
to you to alter some aspects of the GPUSPH run.

\begin{description}
\item [-{}-device \emph{integer}]
Choose which GPU(s) to use for the run.
On the command line: \cmd{./GPUSPH -{}-device N}, where N is the
(integer) number of the device you wish to use. 
To find the number associated with each of your CUDA-enabled 
devices (graphics cards), you
can use the CUDA SDK program DeviceQueryDrv. If you only have one
CUDA-enabled GPU, the only possible choice for N is~0, which is the
default.
If you want to run the simulation on several GPUs, the command is:
\cmd{./GPUSPH -{}-device i,j,k}
\item [-{}-deltap \emph{float}]
Change the resolution (inter-particle spacing) at which the problem
should be run.
\item [-{}-tend \emph{float}]
The model time in seconds when you wish the model to stop.
\item [-{}-dem \emph{string}]
For the Problem TestTopo: the name of the DEM file to use.
\item [-{}-resume \emph{fname}]
Resume from the given file (HotStart file saved by HotWriter).
\item [-{}-checkpoint-every \emph{float}]
HotStart checkpoints will be created every VAL seconds of simulated time (float VAL, 0 disables).
\item [-{}-checkpoints \emph{integer}]
Number of HotStart checkpoints to keep.
\item [-{}-maxiter \emph{integer}]
Break after this many iterations.
\item [-{}-dir \emph{string}]
Use given directory for dumps instead of date-based one.
\item [-{}-nosave]
Disable all file dumps but the last.
\item [-{}-gpudirect]
Enable GPUDirect for RDMA (requires a CUDA-aware MPI library).
\item [-{}-striping]
Enable computation/transfer overlap in multi-GPU (usually convenient for 3+ devices).
\item [-{}-asyncmpi]
Enable asynchronous network transfers (requires GPUDirect and 1 process per device).
\item [-{}-num-hosts \emph{integer}]
Uses multiple processes per node by specifying the number of nodes.
\item [-{}-byslot-scheduling]
MPI scheduler is filling hosts first, as opposite to round robin scheduling.
\item [-{}-debug \emph{flags}]
Enable specified debug flags.
\item [-{}-help]
Show the help and exit.
\end{description}


\section{Installing pre/post processing tools}

\subsection{Installing SALOME}
You can download SALOME from:\\
\url{http://www.salome-platform.org/downloads/current-version}

For this you need to register on the SALOME website.
Then, follow the installation instructions from the SALOME website and the
installer.

\subsection{Installing CRIXUS}
Crixus is a preprocessing tool for GPUSPH. \\

\textbf{Prerequisites}:
\begin{itemize}
\item cmake $\ge$ 2.8
\item cuda
\item hdf5 $\ge$ 1.8.7
\end{itemize}

\textbf{Getting CRIXUS}:
\begin{shellcode}
git clone https://github.com/Azrael3000/Crixus.git
cd Crixus.git
\end{shellcode}

\textbf{Compiling Crixus}: \\
Crixus uses CMake for compilation. 
Let us assume that you have CRIXUS in a \cmd{Crixus.git} directory. 
and you want the building to happen in \cmd{Crixus.git/build} then follow the commands below:
\begin{shellcode}
mkdir build
cd build
cmake ..
make
\end{shellcode}
Note that you should not run cmake in the main Crixus folder.

The binary is then located at \cmd{Crixus.git/build/bin/Release/Crixus}.
Note that "make install" is not supported yet. To easily change the parameters of cmake you can use ccmake instead.

If hdf5 cannot be found due to lacking environmental variable you can edit the main CMakeLists.txt which has a commented line that reads:
\begin{shellcode}
#set(ENV{HDF5_ROOT} "/your/path/to/hdf5")
\end{shellcode}
Uncomment it and set the respective hdf5 path in order to use your custom installation.\\

To finish the installation it is recommended to add the path to the CRIXUS binary to your
\cmd{$PATH} environment variable. Add this line in \cmd{~/.bashrc}:\\
\cmd{export PATH=/your_path/Crixus.git/build/bin/Release/Crixus:\$PATH}\\
where \cmd{/your_path} is your path to the CRIXUS directory.

\subsection{Installing PARAVIEW}

\todo{PARAVIEW is available from the Linux packages.}
