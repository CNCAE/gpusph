To run simulations with your own setup, you must create a new
\cmd{Problem}. This is done by creating a new \cpp\ source file, with
the associated header (e.g.\ \cmd{MyProject.cu} and \cmd{MyProject.h}),
placing them under \cmd{src/problems}, running \cmd{make problem=MyProject} to
build it, and finally \cmd{./GPUSPH} to run it. Beginners should use one
of the provided sample files as a template for their project.
There are two main samples available in the \cmd{src/problems} directory: 
\cmd{XProblemExample} (for Lennard-Jones or dynamic boundaries) 
and \cmd{XCompleteSaExample} (for semi-analytical boundaries).\\

%\cmd{MyProject.cu} should define a new \cpp\ class by the same name
%(\cmd{MyProject}), derived of the \cmd{XProblem} class. The constructor
%for \cmd{MyProject} should set up the domain size, the physical
%parameters to be used in the simulation (gravity, viscosity,
%sound-speed, etc), as well as any other simulation parameter (such as
%SPH formulation to use, viscosity model, boundary type, etc).
%
%
%\todo{Next: sections describing each part of a project file, both the
%\cmd{.cc} source and the \cmd{.h} header, with a step-by-step
%construction.}

\section{Anatomy of a project}

Below are the steps required to build a new project and run it with GPUSPH:
\begin{enumerate}
\item In case you use semi-analytical boundary conditions, follow the steps 
described in section \ref{sec:preprocess_sa} for the pre-processing;
\item Create \cmd{MyProject.cu} and \cmd{MyProject.h} files in 
the \cmd{src/problems} directory
\item In the GPUSPH folder, compile the code for your project:
\begin{shellcode}
make problem=MyProject
\end{shellcode}
\item Execute GPUSPH:
\begin{shellcode}
./GPUSPH
\end{shellcode}
\item Follow the steps described in section \ref{sec:postprocess} 
to visualize and post-process the results.
\end{enumerate}

\section{Specific pre-processing for semi-analytical boundaries}\label{sec:preprocess_sa}

Several types of boundary conditions are available in GPUSPH. 
They are described in section \ref{sec:boundary_conditions}.
With classical boundary conditions (Lennard-Jones, dynamic boundaries) the problem 
geometries are defined and filled with particles by GPUSPH itself.
For simulations involving complex objects and/or open boundaries, 
the semi-analytical boundaries can be used. They make it possible 
to import a geometry in \cmd{h5sph} format. These geometries are 
generated using a mesher (\textit{e.g.} \cmd{SALOME}) and \cmd{CRIXUS},
an open-source software which is able to fill the computational 
domain with particles. 

The pre-processing steps specific to semi-analytical boundaries 
are the following:
\begin{enumerate}
\item create a mesh of the boundaries in SALOME and 
export it as a binary STL file;
\item run the \cmd{testTriangle} script to know the 
minimum interparticle distance to set in the simulation, 
with the following syntax:
\begin{shellcode}
testTriangle NameOfTheSTLFile.stl 0.1
\end{shellcode}
where 0.1 could be any number (the program just checks if 
the distance between the center and the vertex of a 
triangle is bigger than the specified number, and returns 
the maximum and minimum found for that dimension);
\item run CRIXUS to fill the domain with particles:
\begin{shellcode}
Crixus NameOfTheINIFile.ini
\end{shellcode}
\item copy the resulting \cmd{h5sph} file(s) containing 
the initial particles to the directory \cmd{data_files}.
\end{enumerate}

These steps are described with more detail below.

\subsection{Preparing the geometry with SALOME}

The input files for CRIXUS are meshes of:
\begin{itemize}
\item the total domain's boundaries
\item the free-surface
\item the special boundaries (for moving objects 
and/or open boundaries)
\end{itemize}
These meshes must be composed of triangles and can 
be generated using SALOME, which is an open source 
software very useful for 3D modeling and meshing.\\

\textbf{It is important to generate meshes with homogeneous 
triangle size, otherwise the quality of results may be affected.}\\


In order to start a new project in SALOME, click on \cmd{File/New}. 
When you save your project, SALOME creates a file with the 
\cmd{.hdf} extension, which stores all the geometry elements 
and meshes that you design for your project.\\

\textbf{Building the geometry}\\

The complete SALOME documentation for the GEOMETRY module can be found here:\\
\url{http://docs.salome-platform.org/7/gui/GEOM/}\\

Designing is easy in SALOME. To start building the geometry 
elements, click on the “Geometry” module. 
There are 7 types of basic geometrical elements:
\begin{enumerate}
\item VERTEX: it can be created by providing its coordinates, 
by clicking on the vertex of another geometry element, 
by using another point as reference…There are many ways 
which are described in the “Point Construction” window 
which appears when we click on “Create a point”
\item SEGMENT: it can be created providing two points 
that were previously stated, or using the intersection of 
two planar elements. 
\item WIRE: a wire is just a series of segments. 
It can be a closed wire or if the end matches the start, 
or an open one.
\item FACE: a face is just a limited plane 
\item SHELL: a shell is a series of faces. 
SALOME would consider it a closed shell when it 
encloses a volume
\item SOLID: a solid is just a limited part of the 3D space; 
it can be easily created on the basis of a closed shell
\item COMPOUND: a compound is just the combination 
of two or more elements of different type, merged 
into one single element
\end{enumerate}
Apart from these basic geometry types, we can find 
three special types, which are 
DIVIDED DISK, DIVIDED CYLINDER, and SMOOTHING SURFACE. 
Among these three special types, the most interesting 
is the smoothing surface, as it is useful to create 
3D surfaces from a point cloud.
Finally, we can find auxiliary geometry elements such as 
circles, ellipses, arcs, vectors, sketches, polylines, cylinders, 
cones, spheres, cubes, torus, disks, T shape pipes, etc.

SALOME makes it possible to import geometries from 
a wide range of file types: STL, BREP, STEP,etc. 
It is possible to import a geometry in STL format 
(generated with another 3D modeling software, such 
as Autocad, SolidWorks, Catia, etc.). \\

\textbf{Caution}: STL files are ASCII or binary 
files in which geometry is described by triangles. 
Each element of an STL file is composed by the 3 
coordinates of each 3 vertex of the triangle, 
and the 3 components of the triangle’s normal vector. 
This means that when we export some geometry 
elements in STL format, triangles would be automatically created. 
When importing this file in SALOME, the geometry is then 
composed of triangular faces and that the meshing operations 
to be implemented afterwards are influenced by this previous 
and automatic discretization of the geometry.
This results in bad mesh quality.
So when importing geometry on a STL format, a redesigning of it is necessary 
in order to obtain a good mesh quality.\\

Other very useful tools of SALOME are the boolean operations on solids.
It is possible to fuse, intersect solid objets, use a solid object as a 
cutting tool for another one, etc.
It is also possible to perform operations like rotation, translation, etc. on
the geometrical objects.\\

\textbf{Generating the mesh}\\

The complete SALOME documentation for the MESH module can be found here:\\
\url{http://docs.salome-platform.org/latest/gui/SMESH/index.html}\\

Once the geometry is defined, it can be meshed using SALOME.
In order to access the meshing tools, change from the Geometry 
module to the Mesh module.
To create a mesh from a geometry element, click on \cmd{Create Mesh}, 
and a window opens (see Figure \ref{fig:salome_screenshot_1}) in which the following options are available:
\begin{itemize}
\item Name: the name of the mesh which is going to be created
\item Geometry: the geometry element that we want to mesh
\item 3D/2D/1D/0D: it’s the nature of the mesh that we are 
going to create, it automatically chooses the correct one 
depending on the type of element that we have specified in Geometry
\item Algorithm: is the meshing method’s algorithm. 
Netgen 1D-2D works well for shell meshing.
\item Hypothesis: here we can specify the hypothesis to be used 
by the algorithm method. Clicking on the first icon on the right, 
we can specify the parameters of the algorithm. See the SALOME 
documentation for more details about the options. 
For example, for the Netgen 1D-2D algorithm, a window opens 
with all the options shown in the Figure \ref{fig:salome_screenshot_2}.
\end{itemize}

\begin{figure}[h]
  \begin{center}
    \includegraphics[scale=0.7, trim={1820 365 330 170},clip]{fig/salome_screenshot_1.png}
    \caption{Screenshot of the mesh options window in SALOME.}\label{fig:salome_screenshot_1}   
  \end{center}
\end{figure}

\begin{figure}[h]
  \begin{center}
    \includegraphics[scale=0.7, trim={2215 308 50 180},clip]{fig/salome_screenshot_2.png}
    \caption{Screenshot of the Netgen 1D-2D hypothesis window in SALOME.}\label{fig:salome_screenshot_2}   
  \end{center}
\end{figure}

The most relevant mesh options with the Netgen 1D-2D algorithm 
are Max Size and Minimum Size. 
It is important to note that SALOME usually respects the Max Size, 
whereas the minimum size is often ignored due to geometry-mesh 
adaptation problems. In addition, the minimum size would be 
always delimited by the characteristic size of the geometry, 
that is to say, the minimum length of the faces composing the shell. 
Regarding the option Fineness, Fine works usually well. 
Once we press OK and then Apply, an element of mesh type 
will appear in the Object Browser on the left side of the screen. 
The icon will appear with an exclamation mark on it: that means 
that the mesh has not been computed yet. To do so, we click on 
the icon Compute or we just do right click and we select 
the option Compute. The algorithm will now begin iterating 
until a solution has been found. 
The mesh is then prepared to be exported as an STL file 
in order to be used as an input for CRIXUS. \\

\textbf{Caution}: STL files for CRIXUS need to be binary, 
so when exporting the mesh, make sure you choose the correct 
file format at the bottom tab.


\subsection{Check the triangles' dimensions with \cmd{testTriangle}}
In order to know the triangle dimensions that we have just created in SALOME, 
we can run the TESTTRIANGLE script, available with CRIXUS.

To compile TESTTRIANGLE, use this command in the \cmd{Crixus.git/scripts} directory:
\begin{shellcode}
gcc test-triangle-size.c -lm -o testTriangle
\end{shellcode}
It is recommended to add the path to the TESTTRIANGLE binary to your
\cmd{$PATH} environment variable. Add this line in \cmd{~/.bashrc}:
\begin{shellcode}
export PATH=/your_path/Crixus.git/scripts/testTriangle:$PATH
\end{shellcode}
where \cmd{/your_path} is your path to the CRIXUS directory.

To work correctly, the distance between particles (dr) of a given simulation 
should not be less than the maximum distance between the center and the 
vertex of a triangle. This is why TESTTRIANGLE is used to get the maximum 
value of this magnitude for all triangles of the main geometry STL file. 
Run the program with the command:
\begin{shellcode}
./testTriangle NameOfTheSTLFile.stl 0.1
\end{shellcode}
Where 0.1 could be any number. 
The program just checks for each triangle of the mesh whether the 
radius is bigger than the specified number, and returns the 
maximum and the minimum value found for the radius of each triangle. 
The user is supposed to set the dr of the simulation equal 
or superior to the maximum value got by TESTTRIANGLE. 
However, it should not be much bigger, as the number of neighbors 
would increase enormously, driving the simulation slower in terms of computation time.

\subsection{Fill the geometry with particles using CRIXUS}

CRIXUS is an open source software which performs the initialization 
of the fluid as a previous step for GPUSPH. 
As input, it basically needs the STL files describing the model geometry 
and, if necessary, an STL file describing the free-surface and/or the special 
boundary meshes. 
In addition, we have to specify the distance between particles 
of the simulation and other options as described below.\\

\textbf{Remark}: what follows is just a summary of the CRIXUS manual, 
available in the README file of CRIXUS.\\

CRIXUS takes an INI file as an input. 
This is essentially a text file with \cmd{.ini} extension 
where we specify all the necessary STL files and other parameters.
The INI file follows the following structure:
\begin{ccode}
[mesh]
  stlfile=salome_box_0.02_with_floating_box.stl
  dr=0.017634
  fshape=sa_box_fshape.stl
[special_boundary_grids]
  mesh1=sa_box_sbgrid_1.stl
  mesh2=sa_box_sbgrid_2.stl
[fill_0]
  option=geometry
  xseed=0.5
  yseed=0.5
  zseed=0.2
  dr_wall=0.018
[output]
  format=h5sph
  name=xcomplete_sa_example
  split=yes
\end{ccode}

Every section of the file starts with the title \cmd{[XXX]} 
and all following statements refer to that section.
These sections are described below.\\

\textbf{Section [mesh]}\\
Here is where the main geometry STL file will be specified. 
The value of stlfile is the name of the STL file containing 
the geometry of the walls and other elements limiting 
the movement of the fluid. 
The value of dr is the distance between particles 
which CRIXUS will use to fill the domain with fluid.
swap_normals enables the user to change the orientation 
of the shell’s faces of the STL file. Faces should be 
inner-fluid oriented for the simulation to work correctly.
fshape is optional and specifies the free surface of 
the fluid which we want the simulation to start with. 
The geometry of the surface has to be represented by a STL file, 
whose name is the value of fshape. If no initial surface is 
specify, CRIXUS would fill the domain until the Z limit 
is reached. Note that this time the STL file need to be binary, 
but its meshing is irrelevant since its only goal 
is to limit the domain’s filling.\\

\textbf{Section [fill\_number]}\\
In this section we tell CRIXUS how to fill the domain. 
There are two ways of doing that: we can fill the domain by boxes 
(option=box) or by a seed point (option=geometry).  The first one 
just fills the specified box with fluid, whereas the second 
one starts filling the domain from the specified seed point 
and only stops if it reaches a wall or the free surface. 
We can call the filling algorithm as many times as wished, 
even with different filling option. 
This is useful when we want to initialize the fluid 
in two areas which are not self connected. \\

\textbf{Section [special\_boundary\_grids]}\\
Here we are able to specify which boundaries of 
the domain are open boundaries, that is to say, 
boundaries where the fluid can get in and out. 
It is important to note that these boundaries have 
to be part of the geometry represented by the STL 
file specified as stlfile at the beginning. 
Then, the open boundaries will be implemented by specifying 
the names of their STL files.\\

\textbf{Section [output]}\\
In this section we simply choose the format of the output file. 
This could be either \cmd{.h5sph} or \cmd{.vtu} format. 
The first one is a table ready to be used as the $t=0$ 
parameters for the simulation in GPUSPH, whereas the second 
one is the file format that can be read by ParaView. 
In a normal situation, in this section we would 
always write \cmd{format=h5sph}, since this gives us the file 
to be implemented on GPUSPH. Nevertheless, if we want first 
to observe the results of the filling processes in ParaView, 
we would write \cmd{format=vtu}. 
The H5SPH files can be opened in HDF, a program to visualize data in tables.
It is important to know, once again, that these are just the basic options for CRIXUS, 
sufficient to launch most of the simulations. 
It could happen, however, that we needed to specify more options 
in order to customize our filling process: in this case, 
the rest of the information, including the developer’s e-mail 
can be found in the mentioned README file.\\

In order to run CRIXUS, follow these steps:
\begin{enumerate}
\item Open the Linux Terminal
\item Place the directory in the folder where we 
have all the STL files used by CRIXUS, as well as the INI file:
\begin{shellcode}
cd Directory/Of/The/Files
\end{shellcode}
\item Launch Crixus:
\begin{shellcode}
Crixus NameOfTheINIFile.ini
\end{shellcode}
\item Once Crixus has finished, copy the resulting H5SPH files to the \cmd{data\_files} directory of GPUSPH. 
The geometry (i.e., the H5SPH file) is now ready to be used by GPUSPH.
\end{enumerate}

\section{Setting up the simulation}

As said before, the simulation setup only involves manipulating the .cu and the .h 
files of your problem in order to specify all its parameters before running GPUSPH. 

The structure of a problem, is in fact the structure of the .cu file, 
which could be defined as follows:
\begin{enumerate}
\item GEOMETRY. As the mesh geometry has previously been 
created by CRIXUS, we only have to specify the file 
containing this information: the .h5sph file.
\item SIMULATION PARAMETERS. There are several simulation 
parameters that need to be specified, concerning the time, 
the frequency of output writing or specific SPH parameters.
\item INITIAL CONDITIONS. We need to specify an initial 
value for each of the fields to be implemented on each particle.
\item BOUNDARY CONDITIONS. Boundary conditions 
have also to be stated before running GPUSPH.
\end{enumerate}
In the following sections we develop each section to help 
the readers write the .cu file in order to build their own simulation.

There are actually two ways to develop new problems in GPUSPH.  
The first is to modify a test problem that is similar to what you would like to do, 
such as \cmd{WaveTank.cu} or \cmd{DamBreak3d.cu} in the \cmd{src/problems} directory.  
These problems are based on \cmd{Problem.cc}.  
The format of these programs is typically setting up the problem parameters, 
including the fluid and the computational domain, 
filling the appropriate volumes with particles, 
and then copying the particles to the GPU.

The second way is to use the \cmd{XProblem} approach, which is a higher 
level approach requiring less programming and uses \cmd{XProblem.cc} 
instead of \cmd{problem.cc}. This approach is recommended since it is meant to make 
the problem construction easier:
\cmd{XProblem} does much of the work of defining and placing objects of common shape 
(cubes or parallelpipeds, spheres, etc)  in the domain.  
These shapes can be solid or fluid or filled with fluid.  
\cmd{Xproblem} takes care of filling the appropriate volume(s) with particles.   
It has the added feature of taking care of the setup for floating and other moving 
bodies that are handled through a dynamics library:  \cmd{Project Chrono}.  
Further the initialization and copying of the particles to the GPU is built into Xproblem 
(the \cmd{copy_to_array} method is not needed anymore).  

\subsection{XProblem Examples}

There are four supplied XProblem examples in the src/problems directory:\\
\begin{itemize}
\item	\cmd{XBuoyancyTest.cu}
\item	\cmd{XCompleteSaExample.cu}
\item	\cmd{XDamBreak3D.cu}
\item	\cmd{XProblemExample.cu}
\end{itemize}
Each of these examples can be run by typing \cmd{make problem=XBuoyancyTest}, 
for example,  in the top level GPUSPH directory.  
It is recommended that the user try them to ensure 
everything checks out in terms of CUDA and GPUSPH. \\

\cmd{XBuoyancyTest.cu} is the XProblem version of the example problem BuoyancyTest. 
The advantage of the XBuoyancy.cu example is that it has 60\% less lines of code.
The example includes a rectangular tank of still water with a submerged torus 
(or by changing object_type, a cube or sphere) that is released when the problem begins.  
As time advances, the torus rises through the water column as it has a 
density half that of water and then it reaches the free surface and floats.  \\

The output of \cmd{XBuoyancyTest} is written every 0.01 seconds into a file 
in the directory tests designated by the Xproblem name and the date and time.  
The files are in VTU format that can be read by Paraview.  
Alternative formats, such as text, can be chosen by changing the writer in the \cmd{add_writer} command. \\

\cmd{XProblemExample.cu} shows how a matrix of objects can easily be added to a problem.  
The basic problem is a semi-infinite domain with a plane used as a floor (\cmd{addPlane}).  
A 4 x 4 array of solid cubes is set-up using the addCube command multiple times.   
The \cmd{GT_FIXED_BOUNDARY}  (GT=Geometry Type) means that the cubes are solid. 
The cubes are also rotated 45 degrees by a rotate command. 
Then a smaller array of spheres of fluid are defined.  
\cmd{GT_FLUID} is used in the \cmd{addSphere} command.  
Note for the fluid the \cmd{setDensityByMass} establishes the fluid density.  \\

\cmd{XDamBreak3D.cu} is a XProblem version of \cmd{DamBreak3D.cu}, 
but in this case there are 40\%  less lines of code.   
The problem domain is setup using \cmd{makeUniverseBox()}, 
which has as its arguments two opposite corners of the 
project domain—the first corner is the origin. 
This command sets up the domain using analytical planes as boundaries. 
These planes do not require the use of particles.  
Water is added to the domain with the \cmd{addBox()} command -- 
note that the fluid is denoted by  \cmd{GT_FLUID} (GT=GeometryType).  
The fluid behind the dam is 0.4 m deep.  
A variety of obstacles can be added in front of the dam.  
As provided, there is just a single object, but by invoking the model with
 \cmd{./GPUSPH -{}-num_obstacles 3}
three obstacles will be in front of the dam.  
These obstacles can be rotated from their original position 
by \cmd{./GPUSPH -{}-num_obstacles 3 -{}-rotate_obstacle true}\\

Another run-time option includes \cmd{-{}-wet true or false}, 
which puts a $0.1m$ layer of water around the obstacles (and in front of the dam).  


\cmd{XCompleteSaExample.cu} is an example using the Semi-Analytical boundary conditions (SA). 
This type of boundary condition was chosen in the  \cmd{SETUP_FRAMEWORK}, 
which is a class that contains the various simulation choices. 
For example it contains \cmd{boundary<SA_BOUNDARY>} as the choice.   
This example consists of a tank with a free surface and a submerged inlet.  
There is a floating cube as well.  
The example requires data files that are available on the \url{www.gpusph.org} web site:\\
\cmd{wget http://www.gpusph.org/downloads/data_files_XCompleteSaExample.tgz}\\

or, in your browser,\\
\cmd{www.gpusph.org/downloads/data_files_XCompleteSaExample.tgz}\\

The file (\cmd{data_files_XCompleteSaExample.tgz}) is uncompressed 
in the root gpusph directory.  
It will create a directory \cmd{data_files}, containing four \cmd{.h5sph} 
to set up the fluid and boundaries and one \cmd{.stl} file to define the cube.  
(In addition there are five files that were used to generate the input files 
using Crixus, an open source pre-processor).  
The problem is large and will take some time as it involves the semi-analytical boundaries.  
There are 122,642 particles in total, of which 56821 are fluid particles 
and the rest are boundary and vertex particles.  

\subsection{Generic XExample}

To write your own example, you can use one of the examples as a template, 
but they all have a similar format as XExample.  
For example, looking at the text file, \cmd{XBuoyancyTest.cu} in the directory 
\cmd{src/problems}, we see that, after the appropriate \cmd{includes}, 
including \cmd{XBuoyancyTest.h}, the example is defined as a 
child of the \cmd{XProblem} class.  
Then the setup is done following the structure below.


\subsubsection{Framework setup}

The \cmd{SETUP_FRAMEWORK} function enables to change the
 general options of the simulation:
\begin{ccode}
  SETUP_FRAMEWORK(
    kernel<WENDLAND>,
    formulation<SPH_F1>,
    viscosity<DYNAMICVISC>,
    boundary<SA_BOUNDARY>,
    periodicity<PERIODIC_NONE>,
    add_flags<ENABLE_INLET_OUTLET | ENABLE_DENSITY_SUM 
        | ENABLE_MOVING_BODIES | ENABLE_FERRARI>
  );
\end{ccode}

\begin{enumerate}
\item The first item enables to choose the type of kernel:\\
QUADRATIC\\
CUBICSPLINE\\
WENDLAND\\
GAUSSIAN\\
\item The second item enables to choose the type of SPH formulation:\\
SPH\_F1\\
SPH\_F2\\
SPH\_GRENIER\\
where SPH\_F1 is a WCSPH single-fluid formulation, SPH\_F2 is a WCSPH multifluid formulation and
SPH\_GRENIER is another mutli-fluid formulation based on the Grenier formulation.
\item The third parameter is the viscosity model. There are 5 options for this term: \\
ARTVISC\\
KINEMATICVISC\\
DYNAMICVISC\\
SPSVISC\\
KEPSVISC\\
\item The fourth parameter is the type of boundary. There are 4 options for this term:\\
LJ\_BOUNDARY\\
MK\_BOUNDARY\\
DYN\_BOUNDARY\\
SA\_BOUNDARY\\
\item The fifth item makes it possible to enable periodicity:\\
  PERIODIC\_NONE \\
  PERIODIC\_X \\
  PERIODIC\_Y \\
  PERIODIC\_XY \\
  PERIODIC_Z \\
  PERIODIC_XZ \\
  PERIODIC_YZ \\
  PERIODIC_XYZ \\
\item Finally, the \cmd{add_flags} term enables the implementation 
of some extra functions in the simulation, such as the extra Ferrari diffusion term, 
the in \& out boundaries or the waterdepth function, 
which computes the flow depth at a given set of X, Y:\\
ENABLE\_DTADAPT \\
ENABLE\_XSPH \\
ENABLE\_PLANES \\
ENABLE\_DEM \\
ENABLE\_MOVING\_BODIES \\
ENABLE\_INLET\_OUTLET \\
ENABLE\_WATER\_DEPTH \\
ENABLE\_FERRARI \\
ENABLE\_DENSITY\_DIFFUSION \\
ENABLE\_DENSITY\_SUM \\
ENABLE\_GAMMA\_QUADRATURE \\
ENABLE\_INTERNAL\_ENERGY\\
\end{enumerate}

\subsubsection{Generic simulation parameters}

\begin{ccode}
// Initialization of simulation parameters
m_name = "XCompleteSaExample";
set_deltap(0.02f);
physparams()->r0 = m_deltap;
// Set world size and origin.
// HDF5 file loading does not support bounding box 
// detection yet
const double MARGIN = 0.1;
const double INLET_BOX_LENGTH = 0.25;
// size of the main cube, excluding the 
// inlet and any margin
double box_l, box_w, box_h;
box_l = box_w = box_h = 1.0;
// world size
double world_l = box_l + INLET_BOX_LENGTH 
    + 2 * MARGIN; // length is 1 (box) + 0.2 (inlet box length)
double world_w = box_w + 2 * MARGIN;
double world_h = box_h + 2 * MARGIN;
m_origin = make_double3(- INLET_BOX_LENGTH - MARGIN,
    - MARGIN, - MARGIN);
m_size = make_double3(world_l, world_w ,world_h);
// time parameters
simparams()->tend = 40.0;
simparams()->dt = 0.00004f;
simparams()->dtadaptfactor = 0.3;
// open boundary information
simparams()->numOpenBoundaries=2;
\end{ccode}

\begin{itemize}
\item \cmd{m_name} is the problem name.

\item \cmd{deltap} is the distance between particles used in the current simulation. 
For consistency reasons, it has to be the same that we have set in the INI file for CRIXUS.

\item \cmd{m_size} and \cmd{m_origin} are the size and the origin of the domain 
(defined in SALOME in case semi-analytical boundaries are used).

\item \cmd{tend} is the time at which the simulation should stop.

\item \cmd{dt} is the size of the first time-step or the time-step size if the ENABLE\_DTADAPT
flag is not activated.

\item \cmd{dtadaptfactor} is the CFL coefficient, usually taken as 0.3.

\item \cmd{numOpenBoundaries} is the number of open boundaries in the simulation.
\end{itemize}

\subsubsection{SPH parameters}
\begin{ccode}
// Initialization of SPH parameters
simparams()->maxneibsnum = 352;
// buildneibs at every iteration
simparams()->buildneibsfreq = 1;
// Slightly extend kernel radius for gamma computation
simparams()->nlexpansionfactor = 1.1;
// ferrari correction
simparams()->ferrari = 1.0;
\end{ccode}
\begin{itemize}
\item \cmd{maxneibsnum} is simply a limit for the number of neighbors computed 
in the SPH method. It allows the user control the amount of calculus 
done for each particle at each iteration. If the mesh of the geometry 
is coherent with \cmd{deltap}, the maximum neighbor number should not be over 280, 
so setting this number to 300 would be a good choice.
\item \cmd{buildneibsfreq} is the neighbor counting frequency, 
in terms of number of time steps.
\item \cmd{nlexpansionfactor} is the factor increasing the area 
where we count the neighbor particles
for the computation of $\gamma$ with SA boundaries.
\item \cmd{ferrari} is the Ferrari diffusion coefficient, which is used 
to potentiate diffusion dissipation (0 for no extra diffusion, 1 for the maximum)
\end{itemize}

\subsubsection{Physical parameters}
\begin{ccode}
physparams()->gravity = make_float3(0.0, 0.0, -9.81);
size_t water = add_fluid(1000.0);
set_kinematic_visc(0, 1.0e-2f);
set_equation_of_state(water, 7.0f, 50.0f);
\end{ccode} 
This specifies the gravity field, the fluid density 
(through the \cmd{add_fluid} function),
and the equation of state to be used through \cmd{set_equation_of_state}.
In this function, the first argument is the fluid considered,
the second one is the exponent in the equation of state (usually 7),
and the third one is the numerical speed of sound.
The numerical speed of sound can alo be set by specifying reference
velocity and water height:
\begin{ccode}
// Reference quantities for speed of sound computation
setWaterLevel(0.5);
setMaxParticleSpeed(7.0);
\end{ccode} 
   
\subsubsection{Results parameters}
\begin{ccode}
  // Drawing and saving times
  add_writer(VTKWRITER, 1e-1f);
\end{ccode} 
The writer (for the output data) is chosen with the \cmd{add_writer}
command (usually the VTK writer, which provides files to be read by Paraview).  
The file writing frequency (in terms of simulated seconds) can also be specified. 
That is, in this case, we will have a VTU file every 0.1 simulated second for example. 
It is important to note that, since some simulations could become 
too large, this frequency is essential in order to limit the size of the result files.\\


\subsection{Building and initializing the particle system}

\textbf{With classical boundary conditions:}\\

With DYNAMIC or LENNARD-JONES boundary conditions, the
problem geometry and the filling with particles is done
inside GPUSPH. An example of generation of arrays
of cubes and spheres in the computational domain is given in
\cmd{XProblemExample.cu}. 
The geometrical objects can be added using functions like:
\begin{ccode}
addCube(GT_FIXED_BOUNDARY, FT_BORDER,
             Point(X,Y,Z),cube_size);
\end{ccode}
The geometry type (GT) may be fluid, fixed boundary, open boundary, 
floating body, moving body, plane, and testpoint, eg. \cmd{GT_OPENBOUNDARY}. 
They are enumerated in \cmd{XProblem.h}.\\


GPUSPH has a variety of geometrical objects that can be used to generate Problems.
The geometrical objects are defined in the \cmd{src/geometry} folder.
The XProblem class makes it possible to rotate or shift them after they were defined.
They can be assigned a mass and a center of gravity.
In two dimensions, the objects (in \cpp\ terms, classes) include {\em
Point, Vector, Segment, Rect (rectangle), Circle}. In three
dimensions, there are additional objects: {\em Cone, Cube, Cylinder,
Sphere and TopoCube}. Using these objects, many types of Problems can
be constructed. For the three dimensional case, the bottom (
bathymetry) of the problem domain can be input via a file, using the
TopoCube object and a dem file.

The {\em Point} object is usually used as a three dimensional object
containing the location of a point in three dimensions. All numbers are
double precision. Associated with the Point object are functions that
determine distance (or distance squared) of a point from the origin or
the distance from another point.

A {\em Vector} object is a three dimensional double precision object of
three space coordinates, x,y, and z. Vector has a number of associated
and useful functions, such as Vector.norm, for the length of the vector.


The {\em Cube} object is really a parallelepiped, defined by an origin,
given by a Point object, and three vectors are used to define the size
and orientation of the cube. For example, here is a box that delimits
an experimental domain (taken from the DamBreak3D.cc example), called
{\em experiment\_box.} \\

\noindent experiment\_box = Cube(Point(0, 0, 0),Vector(1.6, 0,
0),Vector(0, 0.67, 0), Vector(0, 0, 0.4));\\

This box has a corner located at the origin of the domain, with $(x, y,
z) = (0,0,0)$, and three vectors from this point describe the cube,
which happens to be 1.6 m long in the $x$ direction, 0.67 m long in the
$y$ direction, and $0.4$ in the $z$ direction.

So far we have only defined the cube {\em experiment\-box}, we have
given it no properties. For this particular box, which bounds the
computational domain, its bottom and four sides will be set as boundary
particles, as we will see later.

Associated with the Cube object are commands to fill the inner part of
the box with particles, or to fill the boundaries as with boundary
particles. %Also there are drawing commands for openGL rendering of the cube.


The {\em Cylinder} object is defined by a point that determines the
location of the center of the disk that forms its base, a vector that
defines the radius about the point, and then another vector that defined
the height of the cylinder. The cylinder object also has fill and
FillBorder commands. For example, \\

jet = Cylinder(Point(0.,0.,0.), Vector(0.5,0.,0.), Vector(0.,0.,1.));\\
\\would define a cylinder located at the origin with radius 0.5 and
height 1.0 with the name jet. The Cylinder object can be used to
define a cylindrical column of fluid, using the \verb!jet.Fill!
command for the defined cylinder, jet. The mass of the particles
forming jet is set by \verb!jet.SetPartMass! function. If the jet was
supposed to be a pipe, the \verb!jet.FillBorder!, with suitable
arguments, would use boundary particles for the pipe called jet. Two
of the arguments (Booleans: true or false) of the method determine if
the cylinder is closed on the bottom or the top.

The {\em Sphere} object is defined by a point that determines the center
of the sphere, a vector that determines its radius (and equatorial
normal), and a vector pointing to the sphere's pole. For a sphere,
these two vectors have equal magnitude and are normal to each other.
The Sphere object uses the Circle object in layers to create a sphere.

A {\em TopoCube} object is used to define a domain that has the bottom
of the cube provided by a data file. The geometry of the TopoCube is
determined the same was as in the Cube object. The data file has a
strict format; for example: \\\\ north: 13.2 \\ south: -0.2\\ east:
43.2 \\ west: 0.54 \\ rows: 134\\ cols: 432 \\ \{data in 134 rows
with 432 entries per line; numbers space separated\}\\ \\ The numbers
following the compass directions are the length of the domain described
by the data, in meters. (North and south correspond to the +Y axis and
the -Y axis, while E and W are aligned with the +X and -X directions.)
The internal variables (see problem TestTopo.cc) $nsres$ and $ewres$ are
grid resolutions determined by $nsres= (north-south)/(nrows-1)$ and
$ewres= (east -west)/(ncols-1)$.

The data file is read using the TopoCube.SetCubeDem function, which is
called with arguments (float H, float *dem, int ncols, int nrows, float
nsres, float ewres, bool interpol), where H is the depth of the cube,
*dem points to the array of bathymetric data in the data file, ncols and
nrows are the number of columns and rows in the dem data set, nsres and
ewres is the spacing between the bathymetric data in the north/south
direction and the east/west direction, and interpol (not the police) is
the boolean variable for interpolation. FillBorder will fill a face
with particles--the particular face is determined by face\_num, which
takes on the values of (0,1,2,3), for the front face, the right side
face, the back face, and the left side face (facing the -$x$ direction)
for a rectangular box.

Other objects can be defined and added to the source directory to allow
for additional flexibility.

\textbf{With semi-analytical boundary conditions:}\\

The fluid initialization performed by CRIXUS and stored in the H5SPH 
files is used by GPUSPH to start the simulation. 
The specification of the file containing the fluid particles occurs with the following statement:

\begin{ccode}
addHDF5File(GT_FLUID, Point(0,0,0), 
"./data_files/XCompleteSaExample/0.my_project.fluid.h5sph",
NULL);
\end{ccode}
The specification of the file containing the special boundary particles  occurs with the following statement:\\

\begin{ccode}
// Main container
GeometryID container =
addHDF5File(GT_FIXED_BOUNDARY, Point(0,0,0), 
  "./data_files/MyProject/0.my_project.boundary.kent0.h5sph",
  NULL);
disableCollisions(container);
  
// Inflow boundary 
GeometryID inlet =
  addHDF5File(GT_OPENBOUNDARY, Point(0,0,0), 
  "./data_files/MyProject/0.my_project.boundary.kent1.h5sph",
  NULL);
disableCollisions(inlet);

GeometryID cube =
  addHDF5File(GT_FLOATING_BODY, Point(0,0,0), 
  "./data_files/MyProject/0.my_project.boundary.kent2.h5sph",
  "./data_files/MyProject/MyProject_object_file.stl");
// ouptut forces on the cube
enableFeedback(cube);
// set the cube density
setMassByDensity(cube, 500);

\end{ccode}

In order to specify whether the open boundary is pressure driven or velocity driven, the following lines
are used:
\begin{shellcode}
setVelocityDriven(inlet, VELOCITY_DRIVEN);
setVelocityDriven(inlet, PRESSURE_DRIVEN);
\end{shellcode}

Once again the GT (GeometryType) can be fluid, fixed boundary, open boundary, 
floating body, moving body, plane, or testpoint, eg. \cmd{GT_OPENBOUNDARY}. 

\section{Running your simulation}\label{sec:run}
To run your simulation you first need to compile GPUSPH for your problem.
To do so, in the GPUSPH folder, run:
\begin{shellcode}
make problem=MyProblem
\end{shellcode}
\textbf{Remark}:
\begin{itemize}
\item If you are running a multi-node simulation, do not forget to add the option
\cmd{mpi=1}.
\item If your are running a simulation with moving objects, do not forget to add the
option \cmd{chrono=1}.
\end{itemize}
See the section \ref{sec:compileoptions} or run \cmd{make -{}-help} for the complete liste of compilation options.

\section{Visualizing the results}\label{sec:postprocess}

The results of the simulation are stored in a directory under
\cmd{tests}, named after the used Problem and the date of execution
(e.g. \cmd{tests/DamBreak3D_2014-6-12T13h23}). Data files (found in a
\cmd{data} subdirectory of the specific test directory) are normally
written in VTK Unstructured Grid format (\cmd{.vtu}) and can be
visualized with ParaView.

The files necessary to hotstart the simulation are also stored
in the \\ \cmd{tests/MyProject_2014-6-12T13h23/data} folder.

The run directories and their content are preserved until manually
removed. The \cmd{scripts/rmtests} auxiliary script can be used to clean
up the \cmd{tests} directory.

A tutorial to start using ParaView is available here:\\
\url{http://www.paraview.org/Wiki/Beginning_ParaView}\\

To open a file, click on the first upper icon on the left. 
The VTU files are named as \cmd{PART_00025.vtu} where the number corresponds to
the output files numbering. 
PARAVIEW allows the user to visualize at the same time all the VTU files, 
just clicking on VTUinp.pvd or selecting the all set of \cmd{PART_..vtu} files
(see the Figure \ref{fig:paraview_screenshot_1}).
The set of VTU files can be analyzed as a movie by clicking on 
the play buttons at the top of the screen. 

\begin{figure}[h]
  \begin{center}
    \includegraphics[scale=0.65, trim={1680 335 190 155},clip]{fig/paraview_screenshot_1.png}
    \caption{Screenshot of the window for file opening in PARAVIEW.}\label{fig:paraview_screenshot_1}   
  \end{center}
\end{figure}
After selecting the \cmd{VTUinp.pvd} or the desired \cmd{PART} files, 
a window appears at the bottom-left part of the screen. 
Press \cmd{Apply} to confirm the file opening: the set of particles
appears at the center of the screen.
When pressing \cmd{Apply}, a window opens with three main sections: 
\cmd{Properties}, \cmd{Display} and \cmd{Information}. 
The second one enables the user to decide which field should be printed 
(first tab of the section \cmd{Coloring}). 
With the \cmd{Show} option, we can make the color legend appear, 
and with the \cmd{Edit} one, we can customize it. 
Below this section we find a set of options that enables us to manage the plotting as desired.
The \cmd{Information} section provides information like the total size of the dataset.\\

Below some useful filters of ParaView are listed. The filters are all available 
through the \cmd{Filters} tab at the top of the screen, in the section
\cmd{Alphabetical}, or through shortcuts in the main window.\\
\begin{itemize} 
\item \textbf{Find Data}\\
Find data by scalar value makes it possible to select particles 
on the basis of the value of their fields. When activating a filter
a window opens. In order to select the particles we want, 
in that window (see the Figure \ref{fig:paraview_screenshot_2}),
we have to set the Find tab on Point in that window. 
In the left tab we can choose the desired field, whereas in the right one 
we can set the value (note that there are several conditions: >=, <=, =, between, etc.). 
Once we have set all the options, we press on Run Selection Query and we 
will be able to see a table containing all the particles that match 
the imposed condition. 
In addition; we will be able to see these particles on 
the Layout, colored with the chosen Selection Color.
\begin{figure}[h]
  \begin{center}
    \includegraphics[scale=0.65, trim={1740 300 260 165},clip]{fig/paraview_screenshot_2.png}
    \caption{Screenshot of the \cmd{Find Data} window in PARAVIEW.}\label{fig:paraview_screenshot_2}   
  \end{center}
\end{figure}

\item \textbf{Clip} \\
In order to visualize a section of the domain, 
since the fields are discrete, we are obliged to perform a \cmd{Clip}.
The \cmd{Clip} window is shown in the Figure \ref{fig:paraview_screenshot_3}.  
By changing the \cmd{Clip Type} tab into \cmd{Box}, it is possible to set the dimensions 
and position of the box. It is important to click on the \cmd{Inside Out} button 
to select the particles that are inside the box. Once ready, click on 
\cmd{Apply} to get a new \cmd{Clip} object in the Pipeline Browser.
You can manipulate it in the same way as the main dataset. You can
also perform clips with planes. \\
\begin{figure}[h]
  \begin{center}
    \includegraphics[scale=0.65, trim={1490 160 820 330},clip]{fig/paraview_screenshot_3.png}
    \caption{Screenshot of the \cmd{Clip} window in PARAVIEW.}\label{fig:paraview_screenshot_3}   
  \end{center}
\end{figure}
\textbf{Remark}: the Slice option does not work because the flow fields are not continuous.
To make a slice, we currently apply a thin box-type clip to the dataset.\\

\item Other useful filters are the \textbf{Threshold}, \textbf{Calculator}, \textbf{Scatter Plot}, etc.

\end{itemize}

\textbf{Saving your results} \\

\textbf{Save Data} \\
You can generate a table in CSV format containing the values of the 
fields for each particle for each PART file or filtered dataset. 
If you click on \cmd{File/Save Data}, a window appears where 
you can set the name and other options for the result file. 
There are many options for the format of the file, but the recommended 
one is .csv, as you can visualize it on the Linux LibreOffice Calc 
or Windows Excel. In addition, 
you can change the format of the file to .dat in order to open 
it with a text editor. 

\textbf{Save State} \\
You can save your PARAVIEW postprocessing state in a file by
clicking on \cmd{File/Save State}. The state file is in ascii
format so you can edit it with a text editor.
You can also apply it to other datasets than the original one, 
which is very useful in order to avoid having to repeatedly
perform the same filtering operations.
